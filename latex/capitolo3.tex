\chapter{Prototipo sviluppato}
\label{chap:3}
Nel capitolo precedente sono state esaminate le differenze sostanziali tra un'approccio basato su Rust in combinazione con WebAssmebly e uno basato su Node.js.
In questo capitolo si presenterà il prototipo sviluppato con l'obiettivo di comprendere l'impatto di tali differenze in un'applicazione pratica.
\section{Descrizione dell'applicazione}
Il prototipo sviluppato è un applicazione dedicata all'elaborazione digitale di immagini, concepita per simulare un contesto realistico in cui le operazioni richiedono una considerevole quantità di elaborazioni da parte della CPU.
\\Dato il limitato tempo disponibile, non è stato possibile esplorare a fondo il contesto del \emph{digital image processing}. Sono state invece utilizzate librerie già pronte in entrambe i linguaggi, senza scendere troppo in profondità nella programmazione di basso livello.
\\L'architettura dell'applicazione seguirà un modello client-server per entrambe le implementazioni.
In particolare il cliente sarà responsabile di fornire i file da processare e le relative specifiche sulle modifiche da apportare.
Il servitore eseguirà le modifiche richieste e risponderà al client con il percorso della nuova immagine, la quale sarà pronta per essere scaricata.
\\Nel processo di selezione delle possibili modifiche da apportare, è stato essenziale individuare due librerie nei rispettivi linguaggi utilizzati.
Successivamente, per garantire uniformità nelle opzioni di modifica disponibili, sono state estratte le seguenti funzionalità comuni: 
\begin{itemize}
    \item ridimensionamento;
    \item rotazione di 90°;
    \item ribaltamento in orizzantale;
    \item conversione in bianco e nero;
    \item aumento/diminuzione del contrasto;
    \item aumento/diminuzione di luminosità;
\end{itemize}
Tali operazioni sono state selezionate poiché rappresentano funzionalità frequentemente utilizzate anche da utenti comuni, oltre a caratterizzarsi per la loro eterogeneità. Alcune di queste coinvolgono esclusivamente la manipolazione dei pixel, come ad esempio la rotazione e il ribaltamento, mentre altre, come la conversione in scala di grigi o la modifica del contrasto/luminosità, comportano modifiche dirette sui pixel stessi.
\newpage
\section{Setup sperimentale}
\section{Metodologia}
\section{Implementazione in Rust e Wasm/WASI}
\section{Implementazione in Node.js}
\section{Valutazione delle prestazioni}
\section{Qualità del codice e manutenibilità}
\section{Esperienza di sviluppo}
\section{Scalabilità e concorrenza}
\section{Conclusioni}
\section{Sviluppi futuri}