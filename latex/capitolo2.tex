\chapter{Tecnologie utilizzate}
\label{chap:2}

\section{Introduzione alle tecnologie}
\label{sec:IntroduzioneTecnologie}
Come già introdotto brevemente nel capitolo \ref{chap:1}, per lo scopo di questa tesi si è scelto di confrontare due approcci differenti, ma sempre più utilizzati nelle applicazioni web moderne.
In particolare si partirà da WebAssembly e dall'interfaccia di sistema WASI, per finire con un'introduzione anche su Node.js.

\section{WebAssembly}
\label{sec:Wasm}
WebAssembly (Wasm) è uno standard che definisce un formato binario (.wasm) e un relativo formato testuale (.wat) per la scrittura di codice eseguibile nelle pagine web. 
\\Esso è nato come integrazione a javascript, per consentire l'esecuzione di codice ad una velocità paragonabile a quella del codice nativo.
\textbf{Il codice WebAssembly è eseguito all'interno di una sandbox garantendo così sicurezza.
I programmi possono essere compilati da svariati linguaggi di alto livello in moduli Wasm. rendendo possibile l'esecuzione di applicazioni lato client in maniera che prima era impensabile.} MODIFICARE!!!
Al giorno d'oggi praticamente ogni browser supporta WebAssembly e il suo sviluppo è portato avanti dal \emph{W3C WebAssembly Working Group}

\subsection{Storia e Origini di WebAssembly}
\subsubsection{I predecessori}
NaCl e asm.js
\subsection{Architettura di WebAssembly}
\subsection{Vantaggi di WebAssembly}

\newpage
\section{WebAssembly System Interface}
\label{sec:WASI}
\subsection{Ruolo di WebAssembly System Interface}
\subsection{Rust e WASI}
\subsection{Struttura di WASI}
\subsection{Wasmtime}
\subsection{Applicazioni e Casistiche d'Uso di WASI}

\newpage
\section{Node.js}
\label{sec:Node}
\subsection{Panoramica di Node.js}
\subsection{Vantaggi di Node.js}
\subsection{Ecosistema di Node.js}

\newpage
\section{Confronto tra tecnologie}
\label{sec:Confronto}
\subsection{Prestazioni}
\subsection{Sicurezza}
\subsection{Facilità di Sviluppo}
\subsection{Scalabilità ed Espandibilità}

\newpage
\section{Conclusioni preliminari}
\label{sec:ConclusioniTecnologie}

\newpage
\lstinputlisting[label=lst:hello, firstline=2, lastline=4, caption={I directly included a portion of a file}]{code/hello.py}

\begin{lstlisting}[language=Java, label=lst:java, caption={Some code in another language than the default one}]
public void prepare(AClass foo) {
        AnotherClass bar = new AnotherClass(foo)
}
\end{lstlisting}
