%CLASSE DOCUMENTO - LINGUA E DIMENSIONE FONT
\documentclass[corpo=11pt,numerazioneromana]{toptesi}

%%%%%%%%%%%%%%%%%%%%%%%%%%%%%%%%%%%%%%%%%%%%%%%%%%%%%%%%%%%%%%%

% INCLUSIONE PACCHETTI
\usepackage[classica]{topfront}
\usepackage[utf8]{inputenc} %utf8
\usepackage[italian]{babel}
\usepackage[T1]{fontenc}
\usepackage{blindtext}
\usepackage{graphicx,wrapfig}
\usepackage{booktabs}
\usepackage{lmodern}
\usepackage{varioref}
\usepackage{url}
\usepackage{array}
\usepackage{paralist}{\obeyspaces\global\let =\space}
\usepackage{verbatim}
\usepackage{subfig}
\usepackage{tabularx}
\usepackage{amsmath}
\usepackage{amsfonts}
\usepackage{float}
\usepackage{amssymb}
\usepackage{multicol}
\usepackage{multirow}
\usepackage[pass]{geometry}
\usepackage[figuresright]{rotating}
\usepackage{algorithm}
\usepackage{algorithmic}
\usepackage{amsmath}
\usepackage[babel]{csquotes}
\usepackage{hyperref}
\usepackage[backend=biber,bibencoding=ascii, sorting=none]{biblatex}
\usepackage{listings}
\usepackage{color}
\definecolor{GrayCodeBlock}{RGB}{241,241,241}
\definecolor{BlackText}{RGB}{110,107,94}
\definecolor{RedTypename}{RGB}{182,86,17}
\definecolor{GreenString}{RGB}{96,172,57}
\definecolor{PurpleKeyword}{RGB}{184,84,212}
\definecolor{GrayComment}{RGB}{170,170,170}
\definecolor{GoldDocumentation}{RGB}{180,165,45}

\definecolor{lightgray}{rgb}{.9,.9,.9}
\definecolor{darkgray}{rgb}{.4,.4,.4}
\definecolor{purple}{rgb}{0.65, 0.12, 0.82}


\lstdefinelanguage{rust}
{
    columns=fullflexible,
    keepspaces=true,
    frame=single,
    framesep=0pt,
    framerule=0pt,
    framexleftmargin=4pt,
    framexrightmargin=4pt,
    framextopmargin=5pt,
    framexbottommargin=3pt,
    xleftmargin=4pt,
    xrightmargin=4pt,
    backgroundcolor=\color{GrayCodeBlock},
    basicstyle=\ttfamily\color{BlackText},
    keywords={
        true,false,
        unsafe,async,await,move,
        use,pub,crate,super,self,mod,
        struct,enum,fn,const,static,let,mut,ref,type,impl,dyn,trait,where,as,
        break,continue,if,else,while,for,loop,match,return,yield,in
    },
    keywordstyle=\color{PurpleKeyword},
    ndkeywords={
        bool,u8,u16,u32,u64,u128,i8,i16,i32,i64,i128,f32,f64,char,str,
        Self,Option,Some,None,Result,Ok,Err,String,Box,Vec,Rc,Arc,Cell,RefCell,HashMap,BTreeMap,
        macro_rules
    },
    ndkeywordstyle=\color{RedTypename},
    comment=[l][\color{GrayComment}\slshape]{//},
    morecomment=[s][\color{GrayComment}\slshape]{/*}{*/},
    morecomment=[l][\color{GoldDocumentation}\slshape]{///},
    morecomment=[s][\color{GoldDocumentation}\slshape]{/*!}{*/},
    morecomment=[l][\color{GoldDocumentation}\slshape]{//!},
    morecomment=[s][\color{RedTypename}]{\#![}{]},
    morecomment=[s][\color{RedTypename}]{\#[}{]},
    stringstyle=\color{GreenString},
    string=[b]"
}


\lstdefinelanguage{JavaScript}{
  keywords={typeof, new, true, false, catch, function, return, null, catch, switch, var, const, let, if, in, while, do, else, case, break},
  keywordstyle=\color{PurpleKeyword}\bfseries,
  ndkeywords={class, export, boolean, throw, implements, import, this},
  ndkeywordstyle=\color{darkgray}\bfseries,
  identifierstyle=\color{black},
  sensitive=false,
  comment=[l]{//},
  morecomment=[s]{/*}{*/},
  commentstyle=\color{purple}\ttfamily,
  stringstyle=\color{GreenString}\ttfamily,
  morestring=[b]',
  morestring=[b]"
}

\lstset{
   language=JavaScript,
   backgroundcolor=\color{GrayCodeBlock},
   extendedchars=true,
   basicstyle=\footnotesize\ttfamily,
   showstringspaces=false,
   showspaces=false,
   numbers=left,
   numberstyle=\footnotesize,
   numbersep=9pt,
   tabsize=2,
   breaklines=true,
   showtabs=false,
   captionpos=b
}


%%%%%%%%%%%%%%%%%%%%%%%%%%%%%%%%%%%%%%%%%%%%%%%%%%%%%%%%%%%%%%%

% CONFIGURAZIONE LINK E RIFERIMENTI
\hypersetup{%
    pdfpagemode={UseOutlines},
    bookmarksopen,
    pdfstartview={FitH},
    colorlinks,
    linkcolor={black}, %COLORE DEI RIFERIMENTI AL TESTO
    citecolor={blue}, %COLORE DEI RIFERIMENTI ALLE CITAZIONI
    urlcolor={blue} %COLORI DEGLI URL
}

%%%%%%%%%%%%%%%%%%%%%%%%%%%%%%%%%%%%%%%%%%%%%%%%%%%%%%%%%%%%%%%

% CONFIGURAZIONE LISTATI/CODICE - CANCELLARE SE NON NECESSARIO
% PYTHON - BIANCO E NERO
\lstset{%
	captionpos=b,
	language=Python,
	basicstyle =\small\ttfamily,
	keywordstyle=\color{black}\bfseries,
	breaklines=true,
	breakatwhitespace=true,
	frame=lines,
	numbers=left,
	numberstyle=\footnotesize,
}

\tolerance=1
\emergencystretch=\maxdimen
%\hyphenpenalty=10000
%\hbadness=10000


%%%%%%%%%%%%%%%%%%%%%%%%%%%%%%%%%%%%%%%%%%%%%%%%%%%%%%%%%%%%%%%

% FRENCHSPACING VA _SEMPRE_ ABILITATO PER DOCUMENTI IN ITALIANO
\frenchspacing

%%%%%%%%%%%%%%%%%%%%%%%%%%%%%%%%%%%%%%%%%%%%%%%%%%%%%%%%%%%%%%%

%DEFINIZIONE SEZIONI IN NUMERAZIONE ROMANA
%ELENCO DEI LISTATI/CODICI
\makeatletter
\newcommand\listofcodes{%
 \iffrontmatter\else\frontmattertrue\fi
 \if@openright\cleardoublepage\else\clearpage\fi
 % change the meaning of \chapter in a group
 \begingroup\def\chapter##1{\@schapter}
 \phantomsection % for the hyperlink
 \addcontentsline{toc}{chapter}{Elenco dei listati}
 \lstlistoflistings
 \endgroup
}
\makeatother

\addto\captionsitalian{%
  \renewcommand{\lstlistlistingname}{Elenco dei listati}%
  \renewcommand{\lstlistingname}{Listato}%
}

%%%%%%%%%%%%%%%%%%%%%%%%%%%%%%%%%%%%%%%%%%%%%%%%%%%%%%%%%%%%%%%

% INFORMAZIONI PDF - PERSONALIZZARE
\pdfinfo{%
  /Title    (Esecuzione Efficiente di Web App Rust su Piattaforma Web Assembly)
  /Author   (Davide Crociati)
  /Subject  (WASM/WASI Rust)
  /Keywords (LaTeXi)
}

%%%%%%%%%%%%%%%%%%%%%%%%%%%%%%%%%%%%%%%%%%%%%%%%%%%%%%%%%%%%%%%

% LISTA DEI CAPITOLI DA INCLUDERE - PERSONALIZZARE
\includeonly{
introduzione,
capitolo1,
capitolo2,
capitolo3,
conclusioni
}

% FILE DI BIBLIOGRAFIA
\addbibresource{bibliography.bib}

%%%%%%%%%%%%%%%%%%%%%%%%%%%%%%%%%%%%%%%%%%%%%%%%%%%%%%%%%%%%%%%

% INIZIO DOCUMENTO
\begin{document}

%%%%%%%%%%%%%%%%%%%%%%%%%%%%%%%%%%%%%%%%%%%%%%%%%%%%%%%%%%%%%%%

% UNIVERSITA - NOME
\ateneo{Alma Mater Studiorum - Università di Bologna}

% FACOLTA - NOME
\facolta{Ingegneria}

% CORSO DI LAUREA - NOME
\corsodilaurea{Ingegneria Informatica}

% TIPOLOGIA TESI
\TesiDiLaurea{Tesi di Laurea Triennale}

% TITOLO
\titolo{Esecuzione Efficiente di Web App Rust su Piattaforma Web Assembly}

% SOTTOTITOLO
\sottotitolo{Tesi in Tecnologie Web}

% RELATORE - PROF. NOME E COGNOME
\relatore{prof.\ Paolo Bellavista}

% CANDIDATO - NOME E COGNOME
\candidato{Davide Crociati}

% DATA - MESE ANNO
\sedutadilaurea{Ottobre 2023}

\frontespizio

%%%%%%%%%%%%%%%%%%%%%%%%%%%%%%%%%%%%%%%%%%%%%%%%%%%%%%%%%%%%%%%

%INTERLINEA - DEFAULT 1 
\interlinea{1.4}

%%%%%%%%%%%%%%%%%%%%%%%%%%%%%%%%%%%%%%%%%%%%%%%%%%%%%%%%%%%%%%%

\frontmatter

%%%%%%%%%%%%%%%%%%%%%%%%%%%%%%%%%%%%%%%%%%%%%%%%%%%%%%%%%%%%%%%

% INDICI - ELIMINARE GLI INDICI NON NECESSARI

% INDICE GENERALE
\tableofcontents

% INDICE DELLE FIGURE
\listoffigures

% INDICE DELLE TABELLE
%\listoftables

% INDICE DEI CODICI
\listofcodes
\newpage\null\thispagestyle{empty}\newpage
%%%%%%%%%%%%%%%%%%%%%%%%%%%%%%%%%%%%%%%%%%%%%%%%%%%%%%%%%%%%%%%

\mainmatter

\chapter*{Introduzione}
\addcontentsline{toc}{chapter}{Introduzione} 
\label{Introduzione}
Le moderne applicazioni web offrono un'esperienza utente coinvolgente e altamente interattiva, caratterizzate da funzionalità sempre più complesse e tempi di attesa ridotti al minimo.
\\Tuttavia, questo rapido progresso è stato inevitabilmente accompagnato da un grande aumento del numero di richieste effettuate in rete e dall'utilizzo crescente di risorse computazionali.
\\La maggiore complessità delle applicazioni Web e l’esigenza di offrire agli utenti esperienze interattive sempre più coinvolgenti hanno portato l’ambito dello sviluppo Web a una svolta significativa.
\\L’ottimizzazione delle prestazioni è infatti diventata un aspetto cruciale sia per lo sviluppo server side, che client side e le decisioni prese sin dalla fase di progettazione possono rivelarsi decisive per la buona riuscita di un progetto moderno.
\\Per queste motivazioni in questo lavoro di tesi ci si è posti l'obiettivo di confrontare le soluzioni offerte da due approcci moderni per lo sviluppo Web di applicazioni computazionalmente complesse.
\\In particolare, ci si concentrerà sullo sviluppo server side di un'applicazione per l'elaborazione digitale di immagini.
Tale applicazione verrà sviluppata utilizzando prima il linguaggio \textbf{Javascript} e successivamente \textbf{Rust} in combinazione con \textbf{WebAssembly/WASI}.
\\Si condurrà un'analisi approfondita per comprendere i benefici specifici legati a ciascun approccio, individuando le circostanze in cui uno dei due possa risultare vantaggioso in termini di efficienza computazionale e reattività.
\\Primariamente ci si dedicherà allo studio delle due tecnologie in esame, cercando di individuare le peculiarità e le API necessarie per lo sviluppo di un progetto Web.
\\In seguito verrà eseguita un'analisi comparativa che consentirà di trarre conclusioni preliminari su tali tecnologie e di valutare in modo esaustivo i risultati ottenuti nelle successive fasi di test.
\\Per valutare con precisione l'adeguatezza dei due approcci verrà sviluppato un prototipo che sfrutterà librerie di image processing utilizzando le due tecnologie in esame.
\\Infine verranno eseguiti una serie di test misurando latenza, utilizzo di CPU e consumo di memoria, al fine di verificare i vantaggi e gli svantaggi teorici presentati in precedenza e valutare l'effettiva differenza di reattività per un utilizzatore comune.
\\Si concluderà con un'ampia analisi dei risultati ottenuti e con la verifica del raggiungimento degli obiettivi prefissati.
\\Per garantire una lettura fluida e scorrevole il documento sarà strutturato in capitoli. Nel primo capitolo verranno analizzati i trend di evoluzione del Web, le motivazioni e gli obiettivi della tesi.
Sucessivamente si procederà con l'analisi dettagliata delle tecnologie utilizzate, tramite esempi ed approfondimenti sulle API necessarie.
Si concluderà con il terzo capitolo che presenterà le implementazioni sviluppate e i risultati ottenuti attraverso ciascun approccio analizzato.
Parlare del fantastico wasm? forse si eh
% INCLUSIONE FILE CAPITOLI - PERSONALIZZARE - TENERE COERENTE CON LISTA IN ALTO
\chapter{Introduzione}
\label{chap:1}

\section{Contestualizzazione}
\label{sec:Contestualizzazione}

\subsection{Evoluzione delle applicazioni web}
Inizialmente le applicazioni web erano costituite da semplici pagine statiche contenenti testo e immagini.
Col passare degli anni, grazie all'adozione di JavaScript e di librerie e framework correlati, hanno progressivamente acquisito un carattere più dinamico, con l'introduzione di livelli crescenti di interattività. 
Un cambiamento significativo in tal senso, è avvenuto con l'avvento delle \textbf{Single Page Application (SPA)} e di \textbf{AJAX}.
Tale combinazione, ha infatti introdotto un nuovo paradigma di sviluppo, in cui l'intera appplicazione viene caricata una sola volta e le successive interazioni con l'utente avvengono grazie al caricamento dinamico di contenuti e dati provenienti da un web server, eliminando così l'attesa nel caricamento di una nuova pagina e favorendo l'esperienza d'uso.
\\Parallelamente, la complessità delle funzionalità offerte è cresciuta in modo esponenziale, spaziando da applicazioni di grafica 3D, a simulatori, ad applicazioni di modifica di documenti, immagini e video.
Oggi, è sempre più comune incontrare siti web in grado di gestire complesse operazioni in tempi rapidi, assicurando così quell'interattività alla quale ormai siamo abituati. 
Tuttavia questo progresso è stato accompagnato da un aumento del numero di richieste effettuate in rete e dall'utilizzo intensivo di risorse computazionali, sia lato cliente, che lato servitore.
\begin{figure}
        \begin{center}
                \includegraphics[width=0.9\columnwidth]{images/spa.jpg}
        \end{center}
        \caption{Differenza nel tipo di richieste tra SPA e MPA.}
        \label{fig:spa}
\end{figure}
        
\subsection{Importanza dell'ottimizzazione}
Ad oggi, l'ottimizzazione delle prestazioni è quindi diventata un aspetto cruciale nello sviluppo di applicazioni web.
Gli utenti si aspettano interazioni con bassa latenza, caricamenti rapidi e risposte immediate.
Questa esigenza mette in risalto l'importanza di bilanciare l'aggiunta di funzionalità sofisticate, con l'offerta di una \emph{User Experience} ottimale.
\\I tempi di caricamento prolungati possono portare a un alto tasso di abbandono delle pagine, riducendo l'opportunità di coinvolgere nuovi utenti.
Inoltre, con l'aumentare dell'utilizzo di dispositivi mobili e di conseguenza, di connessioni instabili, l'ottimizzazione diventa ancor più critica per assicurare un'esperienza coerente su diverse piattaforme e condizioni di rete.
Tutto ciò non riguarda solo il lato client, ma coinvolge anche il lato server. 
Un carico eccessivo sui server può influire negativamente sulla scalabilità, causando ritardi nelle risposte e possibili interruzioni del servizio.
L'ottimizzazione deve quindi coinvolgere tutti gli aspetti dell'architettura delle applicazioni web.
\\Nell'implementare ottimizzazioni, sono nati varie soluzioni interessanti. Ad esempio, per gestire task che svolgono molte operazioni di Input/Output si è distinto \textbf{Node.js}, mentre per quanto riguardo l'esecuzione di attività che sfruttano molto la CPU è emerso \textbf{WebAssembly(Wasm)}, in sinergia con l'interfaccia di sistema \textbf{WebAssembly System Interface (WASI)}.


\newpage
\section{Motivazioni e Obiettivi}
\label{sec:Obiettivi}
La crescente complessità delle applicazioni web e l'esigenza di offrire agli utenti esperienze interattive sempre più coinvolgenti hanno portato l'ambito dello sviluppo web a una svolta significativa.
Le aspettative degli utenti si sono evolute verso applicazioni che offrano prestazioni reattive, interattività immediata e funzionalità avanzate.
\\È proprio questo insieme di aspettative a essere alla base delle motivazioni che hanno guidato la scelta del tema di questa tesi di laurea.
\\In particolare, la presente ricerca, si propone di confrontare in modo dettagliato, due differenti approcci di sviluppo per un'applicazione con funzionalità fortemente CPU-Intensive.
\subsection{Tipologia di applicazione}
Per tale confronto, si è optato per una web-app che utilizzi alcune tecniche di elaborazione di immagini.
In particolare, all'utente sarà consentito eseguire l'upload di immagini su un server e indicare una serie di modifiche da apportare (ad esempio "ridimensionamento del 50\%"). Tale server le eseguirà secondo i parametri ricevuti e infine restituirà al cliente le immagini modificate.
\\Non verrà esplorato in modo approfondito il campo dell'elaborazione digitale di immagini, ma ci si limiterà all'implementazione di funzionalità usate spesso da utenti comuni, come ad esempio, ridimensionamento, rotazione, aumento/diminuzione di luminosità e contrasto e altre che verranno specificate nel capitolo \ref{chap:3}.
\\Tale tipologia di applicazione, si sposa bene per lo scopo finale della tesi: valutare come due aprrocci (e due linguaggi) piuttosto differenti, ma sempre più diffusi al giorno d'oggi, risolvano il problema di un'applicazione web che svolga operazioni dall'alto costo computazionale.
\subsection{Metodologie confrontate}
Le due modalità di sviluppo in esame riguarderanno l'utilizzo delle tecnologie JavaScript e WebAssembly server-side e quindi rispettivamente del runtime environment Node.js e del linguaggio di programmazione Rust in combinazione con WebAssembly System Interface. 
\\La scelta di Node.js deriva dalla sua crescente popolarità dovuta all'utilizzo del linguaggio JavaScript, dall'approccio asincrono nella gestione delle richieste e dalla sua ottima scalabilità per applicazioni fortemente File-System-Intensive.
Per quanto riguarda la seconda tecnologia si è invece optato per Rust, in quanto consente sia la scrittura di moduli che succesivamente compilabili in WebAssembly, sia il loro utilizzo efficiente all'interno del codice, mantenendo in questo modo un'ottima coerenza e risultando, teoricamente, una buona scelta per lo sviluppo di applicazioni computazionalmente complesse. 
\begin{figure}
        \begin{center}
                \includegraphics[width=0.7\columnwidth]{images/imageProc.jpg}
        \end{center}
        \caption{Il campo dell'elaborazione digitale di immagini è molto ampio e sarebbe possibile integrare anche operazioni avanzate (edge detection, pattern recognition etc.), che richiederebbero risorse computazionali molto elevate. Per lo scopo di questa tesi ci si limiterà a elaborazioni più semplici, ma in ogni caso rilevanti e sufficienti a mettere alla prova la CPU.}
        \label{fig:imageProc}
\end{figure} 
\subsection{Analisi Comparativa delle Tecnologie}
Un elemento iniziale di questa ricerca sarà un analisi dettagliata delle tecnologie prese in esame. 
Inizialmente, nel capitolo \ref{chap:2}, verrà svolta un'analisi comparativa delle due tecnologie, presentando i vantaggi e gli svantaggi teorici di entrambe.
\\Sarà infatti fondamentale comprendere come ciascuna affronti la complessità legata ad operazioni I/O-intensive e CPU-intensive, per valutare nel modo più opportuno i risultati ottenuti in fase di test e benchmark dell'applicazione sviluppata.
\\Si procederà poi ad illustrare in modo approfondito il funzionamento delle API sfruttate. Verranno analizzate le peculiarità del linguaggio Rust che lo rendono adatto ad applicazioni ad alta intensità computazionale, nonchè l'efficacia di WebAssembly nell'esecuzione di codice di basso livello con prestazioni paragonabili a quelle dei linguaggi nativi.
\\Nel contempo si analizzerà anche la metodologia basata su Node.js, concentrandosi sull'efficienza e la flessibilità che questo ambiente di esecuzione JavaScript può offrire in ambito web.
\\Si proseguirà affrontando un'analisi dettagliata delle prestazioni di ciascuna tecnologia, evidenziando scenari in cui una risulti più vantaggiosa. Questo approfondimento sarà alla base delle successive valutazioni sulle prestazioni delle applicazioni sviluppate con i due metodi presi in esame.
\begin{figure}
        \centering
        \begin{minipage}{0.40\textwidth}
            \centering
            \includegraphics[width=0.9\textwidth]{images/rustwasm.jpg} 
            \caption{Rust e Wasm}
        \end{minipage}\hfill
        \begin{minipage}{0.40\textwidth}
            \centering
            \includegraphics[width=0.9\textwidth]{images/node.png} 
            \caption{Node.js}
        \end{minipage}
    \end{figure}

\subsection{Valutazione dell'Impatto di Wasm}
WebAssembly, in particolare, è emerso come un'innovazione cruciale nel mondo dello sviluppo web.
Consentendo l'esecuzione di codice a basso livello con prestazioni paragonabili a quelle dei linguaggi nativi, Wasm offre la possibilità di ottenere prestazioni elevate all'interno di un ambiente browser-based.
\\Parallelamente, WebAssembly System Interface (WASI) gioca un ruolo chiave nell'estendere il potenziale di WebAssembly. WASI fornisce un'interfaccia standardizzata per l'accesso a risorse di sistema, consentendo alle applicazioni di interagire con l'ambiente circostante in modo controllato e sicuro. Tale capacità è particolarmente rilevante nell'ambito delle applicazioni web CPU-intensive, in quanto consente di accedere, in maniera efficiente, alle risorse di sistema necessarie per eseguire complesse operazioni di calcolo e manipolazione dei dati.

\subsection{Obiettivi}
Si intende esplorare le opportunità offerte dalle tecnologie enunciate sopra, nell'ottica di un'ottimizzazione delle prestazioni.
In questo contesto l'obiettivo sarà quello di comprendere i benefici specifici di ciascun approccio, individuando le circostanze in cui uno dei due possa risultare vantaggioso in termini di efficienza computazionale e reattività. Inoltre si vorrà sottolineare l'importanza, sin dalla fase di progettazione, di un'attenta analisi per determinare se un'applicazione sia maggiormente orientata al calcolo intensivo o all'uso massiccio del File System. 
Per raggiungere in maniera efficace e misurabile gli obiettivi, si sfrutterà un'analisi approfondita delle performance e delle prestazioni delle due applicazioni sviluppate.

\newpage
\section{Trend di evoluzione del web}
\label{sec:Trend}
\subsection{Paradigmi di Sviluppo Moderni}

\subsection{Esperienze Utente Migliorate ?}



\section{Struttura della tesi}
\label{sec:struttura}

This is a reference to a chapter \ref{chap:1}. This is a reference to a figure \ref{fig:doge}. This is a reference to some code \ref{lst:hello}. This is a citation \cite{famous:paper}.

\lstinputlisting[label=lst:hello, firstline=2, lastline=4, caption={I directly included a portion of a file}]{code/hello.py}

\begin{lstlisting}[language=Java, label=lst:java, caption={Some code in another language than the default one}]
public void prepare(AClass foo) {
        AnotherClass bar = new AnotherClass(foo)
}
\end{lstlisting}


\begin{figure}
\begin{center}
\includegraphics[width=0.3\columnwidth]{images/doge.png}
\end{center}
\caption{This is not a figure. It's a caption.}
\label{fig:doge}
\end{figure}

\chapter{Tecnologie utilizzate}
\label{chap:2}

\section{Introduzione alle tecnologie}
\label{sec:IntroduzioneTecnologie}
Come già introdotto brevemente nel capitolo \ref{chap:1}, per lo scopo di questa tesi si è scelto di confrontare due approcci differenti, ma sempre più utilizzati nelle applicazioni web moderne.
In particolare si partirà da WebAssembly e dall'interfaccia di sistema WASI, per finire con un'introduzione anche su Node.js.

\section{WebAssembly}
\label{sec:Wasm}
WebAssembly (Wasm) è uno standard che definisce un formato binario (.wasm) e un relativo formato testuale (.wat) per la scrittura di codice eseguibile nelle pagine web. 
\\Esso è nato come integrazione a javascript, per consentire l'esecuzione di codice ad una velocità paragonabile a quella del codice nativo.
\textbf{Il codice WebAssembly è eseguito all'interno di una sandbox garantendo così sicurezza.
I programmi possono essere compilati da svariati linguaggi di alto livello in moduli Wasm. rendendo possibile l'esecuzione di applicazioni lato client in maniera che prima era impensabile.} MODIFICARE!!!
Al giorno d'oggi praticamente ogni browser supporta WebAssembly e il suo sviluppo è portato avanti dal \emph{W3C WebAssembly Working Group}

\subsection{Storia e Origini di WebAssembly}
\subsubsection{I predecessori}
NaCl e asm.js
\subsection{Architettura di WebAssembly}
\subsection{Vantaggi di WebAssembly}

\newpage
\section{WebAssembly System Interface}
\label{sec:WASI}
\subsection{Ruolo di WebAssembly System Interface}
\subsection{Rust e WASI}
\subsection{Struttura di WASI}
\subsection{Wasmtime}
\subsection{Applicazioni e Casistiche d'Uso di WASI}

\newpage
\section{Node.js}
\label{sec:Node}
\subsection{Panoramica di Node.js}
\subsection{Vantaggi di Node.js}
\subsection{Ecosistema di Node.js}

\newpage
\section{Confronto tra tecnologie}
\label{sec:Confronto}
\subsection{Prestazioni}
\subsection{Sicurezza}
\subsection{Facilità di Sviluppo}
\subsection{Scalabilità ed Espandibilità}

\newpage
\section{Conclusioni preliminari}
\label{sec:ConclusioniTecnologie}

\newpage
\lstinputlisting[label=lst:hello, firstline=2, lastline=4, caption={I directly included a portion of a file}]{code/hello.py}

\begin{lstlisting}[language=Java, label=lst:java, caption={Some code in another language than the default one}]
public void prepare(AClass foo) {
        AnotherClass bar = new AnotherClass(foo)
}
\end{lstlisting}

\chapter{Prototipo sviluppato}
\label{chap:3}
Nel capitolo precedente sono state esaminate le differenze sostanziali tra un'approccio basato su Rust in combinazione con WebAssmebly e uno basato su Node.js.
In questo capitolo si presenterà il prototipo sviluppato con l'obiettivo di comprendere l'impatto di tali differenze in un'applicazione pratica.
\section{Descrizione dell'applicazione}
Il prototipo sviluppato è un applicazione dedicata all'elaborazione digitale di immagini, concepita per simulare un contesto realistico in cui le operazioni richiedono una considerevole quantità di elaborazioni da parte della CPU.
\\Dato il limitato tempo disponibile, non è stato possibile esplorare a fondo il contesto del \emph{digital image processing}. Sono state invece utilizzate librerie già pronte in entrambe i linguaggi, senza scendere troppo in profondità nella programmazione di basso livello.
\\L'architettura dell'applicazione seguirà un modello client-server per entrambe le implementazioni.
In particolare il cliente sarà responsabile di fornire i file da processare e le relative specifiche sulle modifiche da apportare.
Il servitore eseguirà le modifiche richieste e risponderà al client con il percorso della nuova immagine, la quale sarà pronta per essere scaricata.
\\Nel processo di selezione delle possibili modifiche da apportare, è stato essenziale individuare due librerie nei rispettivi linguaggi utilizzati.
Successivamente, per garantire uniformità nelle opzioni di modifica disponibili, sono state estratte le seguenti funzionalità comuni: 
\begin{itemize}
    \item ridimensionamento;
    \item rotazione di 90°;
    \item ribaltamento in orizzantale;
    \item conversione in bianco e nero;
    \item aumento/diminuzione del contrasto;
    \item aumento/diminuzione di luminosità;
\end{itemize}
Tali operazioni sono state selezionate poiché rappresentano funzionalità frequentemente utilizzate anche da utenti comuni, oltre a caratterizzarsi per la loro eterogeneità. Alcune di queste coinvolgono esclusivamente la manipolazione dei pixel, come ad esempio la rotazione e il ribaltamento, mentre altre, come la conversione in scala di grigi o la modifica del contrasto/luminosità, comportano modifiche dirette sui pixel stessi.
\newpage
\section{Setup sperimentale}
\section{Metodologia}
\section{Implementazione in Rust e Wasm/WASI}
\section{Implementazione in Node.js}
\section{Valutazione delle prestazioni}
\section{Qualità del codice e manutenibilità}
\section{Esperienza di sviluppo}
\section{Scalabilità e concorrenza}
\section{Conclusioni}
\section{Sviluppi futuri}
\newpage\null\thispagestyle{empty}\newpage
%\appendix
% INCLUSIONE APPENDICI - - PERSONALIZZARE - TENERE COERENTE CON LISTA IN ALTO
\chapter*{Conclusioni}
\addcontentsline{toc}{chapter}{Conclusioni} 
\label{Conclusioni}
- qualche riga di intro al contesto
\\- mezza pagina di risultati raggiunti
\\- mezza pagina di pro/contro delle tecnologie utilizzate e delle attività svolte
\\- qualche riga di descrizione di possibile lavoro futuro.
\newpage\null\thispagestyle{empty}\newpage

%%%%%%%%%%%%%%%%%%%%%%%%%%%%%%%%%%%%%%%%%%%%%%%%%%%%%%%%%%%%%%%

% RINGRAZIAMENTI - PERSONALIZZARE
\ringraziamenti
Desidero esprimere la mia profonda gratitudine a coloro che mi hanno sostenuto e incoraggiato lungo questo percorso accademico, contribuendo in modo significativo al raggiugimento di questo traguardo.
\\Innanzitutto, desidero ringraziare i miei genitori per il loro sostegno incondizionato e la costante fiducia che hanno riposto in me. Senza di loro, tutto ciò non sarebbe stato possibile.
\\Un ringraziamento speciale va alla mia fidanzata Chiara, per essere stata sempre presente e avermi sostenuto in ogni momento, offrendomi il supporto morale di cui avevo bisogno per continuare.
\\Voglio inoltre ringraziare i miei compagni di corso, Edo, Ric(Bova.) e Lori, per la loro preziosa collaborazione, l'amicizia e il supporto reciproco durante gli ultimi tre anni.
\\Grazie a tutti coloro che hanno contribuito in modo diretto o indiretto a questa tesi.



%%%%%%%%%%%%%%%%%%%%%%%%%%%%%%%%%%%%%%%%%%%%%%%%%%%%%%%%%%%%%%%

% BIBLIOGRAFIA
\phantomsection
\addcontentsline{toc}{chapter}{\refname}
\nocite{*}
\printbibliography

\end{document}