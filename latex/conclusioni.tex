\chapter*{Conclusioni}
\addcontentsline{toc}{chapter}{Conclusioni} 
\label{Conclusioni}
Dopo aver raccolto tutti i dati necessari dai test effettuati è stato possbile analizzarli e verificare se i vantaggi presentati dalle tecnologie sono riscontrati anche nelle implementazioni sviluppate.
\\Come introdotto in precedenza, sono stati effettuati test su tempo di latenza, utilizzo di CPU e consumo di memoria utilizzando una singola elaborazione (resize del 10\%) su un set di tre immagini con dimensione crescente.
\\Per quanto riguarda il tempo di latenza i due approcci hanno restituito risultati simili solamente per richieste con l'immagine con dimensione minore. Esaminando i risultati per le altre due immagini WebAssembly ha distaccato notevolmente Javascript, arrivando ad ottenere un tempo latenza minore del 60\% per l'elaborazione dell'immagine più grande.
\\Successivamente si ha analizzato l'utilizzo di CPU.
\bigbreak
- qualche riga di intro al contesto
\\- mezza pagina di risultati raggiunti
\\- mezza pagina di pro/contro delle tecnologie utilizzate e delle attività svolte
\\- qualche riga di descrizione di possibile lavoro futuro.