\chapter*{Introduzione}
\addcontentsline{toc}{chapter}{Introduzione} 
\label{Introduzione}
Le moderne applicazioni web offrono un'esperienza utente coinvolgente e altamente interattiva, caratterizzate da funzionalità sempre più complesse e tempi di attesa ridotti al minimo.
\\Tuttavia, questo rapido progresso è stato inevitabilmente accompagnato da un grande aumento del numero di richieste effettuate in rete e dall'utilizzo crescente di risorse computazionali.
\\La maggiore complessità delle applicazioni Web e l’esigenza di offrire agli utenti esperienze interattive sempre più coinvolgenti hanno portato l’ambito dello sviluppo Web a una svolta significativa.
\\L’ottimizzazione delle prestazioni è infatti diventata un aspetto cruciale sia per lo sviluppo server side, che client side e le decisioni prese sin dalla fase di progettazione possono rivelarsi decisive per la buona riuscita di un progetto moderno.
\\Per queste motivazioni in questo lavoro di tesi ci si è posti l'obiettivo di confrontare le soluzioni offerte da due approcci moderni per lo sviluppo di applicazioni Web computazionalmente complesse.
\\In particolare, ci si concentrerà sullo sviluppo server side di un'applicazione per l'elaborazione digitale di immagini.
Tale applicazione verrà sviluppata utilizzando prima il linguaggio \textbf{Javascript} e successivamente \textbf{Rust} in combinazione con \textbf{WebAssembly/WASI}.
\\Si condurrà un'analisi approfondita per comprendere i benefici specifici legati a ciascun approccio, individuando le circostanze in cui uno dei due possa risultare vantaggioso in termini di efficienza computazionale e reattività.
\\Primariamente ci si dedicherà allo studio delle due tecnologie in esame, cercando di individuare le peculiarità e le API necessarie per lo sviluppo di un progetto Web.
\\In seguito verrà eseguita un'analisi comparativa che consentirà di trarre conclusioni preliminari su tali tecnologie e di valutare in modo esaustivo i risultati ottenuti nelle successive fasi di test.
\\Per valutare con precisione l'adeguatezza dei due approcci verrà sviluppato un prototipo che sfrutterà librerie di image processing utilizzando le due tecnologie in esame.
\\Infine verranno eseguiti una serie di test misurando latenza, utilizzo di CPU e consumo di memoria, al fine di verificare i vantaggi e gli svantaggi teorici presentati in precedenza e valutare l'effettiva differenza di reattività per un utilizzatore comune.
\\Si concluderà con un'ampia analisi dei risultati ottenuti e con la verifica del raggiungimento degli obiettivi prefissati. In particolare si analizzerà il comportamento di Wasm valutando la sua efficacia in applicazioni CPU-intensive.
\\Per quanto riguarda il seguente documento, nel primo capitolo verranno analizzati i trend di evoluzione del Web, le motivazioni e gli obiettivi della tesi.
Sucessivamente si procederà con l'analisi dettagliata delle tecnologie utilizzate, tramite esempi ed approfondimenti sulle API necessarie, per poi concludere con il terzo capitolo che presenterà le implementazioni sviluppate e i risultati ottenuti attraverso ciascun approccio analizzato.